\chapter{Introduction}

\section{What This Book Is About}
\label{sec:thisBook}

A data graphic is not only a static image but also tells a story about
the data. It activates cognitive processes that are able to detect
patterns and discover information not readily available with the raw
data. This is particularly true for time series, spatial, and
space-time datasets.

There are several excellent books about data graphics and visual
perception theory, with guidelines and advice for displaying
information, including visual examples. Let's mention \emph{The
  Elements of Graphical Data} \cite{Cleveland1994} and
\emph{Visualizing Data} \cite{Cleveland1993} by W. S. Cleveland,
\emph{Envisioning Information} \cite{Tufte1990} and \emph{The Visual
  Display of Quantitative Information} \cite{Tufte2001} by E. Tufte,
\emph{The Functional Art} by A. Cairo \cite{Cairo2012}, and
\emph{Visual Thinking for Design} by C. Ware \cite{Ware2008}.
Ordinarily, they do not include the code or software tools to produce
those graphics.

On the other hand, there is a collection of books that provides code
and detailed information about the graphical tools available with
\textsf{R}. Commonly they do not use real data in the examples and do
not provide advice for improving graphics according to visualization
theory. Three books are the unquestioned representatives of this
group: \emph{R Graphics} by P. Murrell \cite{Murrell2011},
\emph{Lattice: Multivariate Data Visualization with R} by D. Sarkar
\cite{Sarkar2010}, and \emph{ggplot2: Elegant Graphics for Data Analysis} by
H. Wickham \cite{Wickham2009}.

This book proposes methods to display time series, spatial, and
space-time data using \textsf{R}, and aims to be a synthesis of both
groups providing code and detailed information to produce high-quality
graphics with practical examples.

\section{What You Will \emph{Not} Find in This Book}
\label{sec:thisBookIsNot}

\begin{itemize}
\item \textbf{This is not a book to learn \textsf{R}}. 

  Readers should have a fair knowledge of programming with \textsf{R}
  to understand the book. In addition, previous experience with the
  \texttt{zoo}, \texttt{sp}, \texttt{raster}, \texttt{lattice},
  \texttt{ggplot2}, and \texttt{grid} packages is helpful.

  If you need to improve your \textsf{R} skills, consider these
  information sources:

  \begin{itemize}
  \item Introduction to \textsf{R}\footnote{\url{http://cran.r-project.org/doc/manuals/R-intro.html}}
  \item Official manuals\footnote{\url{http://cran.r-project.org/manuals.html}}
  \item Contributed documents\footnote{\url{http://cran.r-project.org/other-docs.html}}
  \item Mailing lists\footnote{\url{http://www.r-project.org/mail.html}}
  \item R-bloggers\footnote{\url{http://www.r-bloggers.com}}
  \item Books related to
    \textsf{R}\footnote{\url{http://www.r-project.org/doc/bib/R-books.html}},
    and particularly \emph{Software for Data Analysis} by John
    M. Chambers \cite{Chambers2008}.
  \end{itemize}


\item \textbf{This book does not provide an exhaustive collection of
  visualization methods.}

Instead, it illustrates what I found to be the most useful and
effective methods. Notwithstanding, each part includes a section
titled ``Further Reading'' with bibliographic proposals for additional
information.


\item \textbf{This book does not include a complete review or
    discussion of \textsf{R} packages.}

  Their most useful functions, classes, and methods regarding data and
  graphics are outlined in the introductory chapter of each part, and
  conveniently illustrated with the help of examples. However, if you
  need detailed information about a certain aspect of a package, you
  should read the correspondent package manual or vignette. Moreover,
  if you want to know additional alternatives, you can navigate through
  the CRAN Task Views about Time
  Series\footnote{\url{http://cran.r-project.org/web/views/TimeSeries.html}},
  Spatial
  Data\footnote{\url{http://cran.r-project.org/web/views/Spatial.html}},
  Spatiotemporal
  Data\footnote{\url{http://cran.r-project.org/web/views/SpatioTemporal.html}},
  and
  Graphics\footnote{\url{http://cran.r-project.org/web/views/Graphics.html}}.


\item \textbf{Finally, this book is not a handbook of data analysis,
    geostatistics, point pattern analysis, or time series theory.}

  Instead, this book is focused on the exploration of data with visual
  methods, so it may be framed in the Exploratory Data Analysis
  approach. Therefore, this book may be a useful complement for superb
  bibliographic references where you will find plenty of information
  about those subjects. For example, \cite{Chatfield2003},
  \cite{Cressie.Wikle2011}, \cite{Slocum.McMaster.ea2005} and
  \cite{Bivand.Pebesma.ea2008}.

\end{itemize}


\section{How to Read This Book}
\label{sec:how-read}

This book is organized into three parts, each devoted to different
types of data. Each part comprises several chapters according to the
various visualization methods or data characteristics. The chapters
are structured as independent units so readers can jump directly to
a certain chapter according to their needs. Of course, there are
several dependencies and redundancies between the sets of chapters
that have been conveniently signaled with cross-references.

The content of each chapter illustrates how to display a dataset
starting with an easy and direct approach. Often this first result
is not entirely satisfactory so additional improvements are
progressively added. Each step involves additional complexity
which, in some cases, can be overwhelming during a first
reading. Thus, some sections, marked with the sign \floweroneleft,
can be safely skipped for later reading.

Although I have done my best to help readers understand the methods
and code, you should not expect to understand it after one
reading. The key is practical experience, and the best way is to try
out the code with the provided data \textbf{and} modify it to suit
your needs with your own data. There is a website and a code
repository to help you in this task.

\subsection{Website and Code Repository}
\label{sec:github}

The book website with the main graphics of this book is located at

\begin{center}
  \url{http://oscarperpinan.github.com/spacetime-vis/}
\end{center}

The full code is freely available from the repository:

\begin{center}
  \url{https://github.com/oscarperpinan/spacetime-vis}
\end{center}

On the other hand, the datasets used in the examples are either
available at the repository or can be freely obtained from other
websites. It must be underlined that the combination of code and data
freely available allows this book to be fully reproducible.

I have chosen the datasets according to two main criteria:
\begin{enumerate}
\item They are freely available without restrictions for public use.
\item They cover different scientific and professional fields
  (meteorology and climate research, economy and social sciences,
  energy and engineering, environmental research, epidemiology, etc.).
\end{enumerate}

The repository and the website can be downloaded as compressed
files\footnote{Repository:
  \url{https://github.com/oscarperpinan/spacetime-vis/archive/master.zip},
  Website:
  \url{https://github.com/oscarperpinan/spacetime-vis/archive/gh-pages.zip}},
and if you use \texttt{git}, you can clone the repository with

\lstset{language=bash,numbers=none}
\begin{lstlisting}
git clone https://github.com/oscarperpinan/spacetime-vis.git
\end{lstlisting}



\section{\textsf{R} Graphics}
\label{sec:r-graphics}

\index{Packages!grid@\texttt{grid}}
There are two distinct graphics systems built into \textsf{R},
referred to as traditional and grid graphics. Grid graphics are
produced with the \texttt{grid} package \cite{Murrell2011}, a flexible
low-level graphics toolbox. Compared with the traditional graphics
model, it provides more flexibility to modify or add content to an
existent graphical output, better support for combining different
outputs easily, and more possibilities for interaction. All the
graphics in this book have been produced with the grid graphics model.

Other packages are constructed over it to provide high-level
functions, most notably the \texttt{lattice} and \texttt{ggplot2}
packages.

\subsection{lattice}
\label{sec:lattice}

\index{Packages!lattice@\texttt{lattice}}

The \texttt{lattice} package \cite{Sarkar2010} is an independent
implementation of Trellis graphics, which were mostly influenced by
\emph{The Elements of Graphing Data} \cite{Cleveland1994}. Trellis
graphics often consist of a rectangular array of panels. The
\texttt{lattice} package uses a \emph{formula} interface to define the
structure of the array of panels with the specification of the
variables involved in the plot. The result of a \texttt{lattice}
high-level function is a \texttt{trellis} object.

For bivariate graphics, the formula is generally of the form
\lstinline{y ~ x} representing a single panel plot with \texttt{y} versus
\texttt{x}. This formula can also involve expressions. The main
function for bivariate graphics is \texttt{xyplot}.

Optionally, the formula may be \lstinline{y ~ x | g1 * g2} and \texttt{y}
is represented against \texttt{x} conditional on the variables
\texttt{g1} and \texttt{g2}. Each unique combination of the levels of
these conditioning variables determines a subset of the variables
\texttt{x} and \texttt{y}. Each subset provides the data for a single
panel in the Trellis display, an array of panels laid out in columns,
rows, and pages.

For example, in the following code, the variable \texttt{wt} of the
dataset \texttt{mtcars} is represented against the \texttt{mpg}, with
a panel for each level of the categorical variable \texttt{am}. The
points are grouped by the values of the \texttt{cyl} variable.

\lstset{language=R,numbers=none}
\begin{lstlisting} 
xyplot(wt ~ mpg | am, data = mtcars, groups = cyl)
\end{lstlisting}

For trivariate graphics, the formula is of the form 
\lstinline{z ~ x * y}, where \texttt{z} is a numeric response, 
and \texttt{x} and \texttt{y} are numeric values evaluated on a
rectangular grid. Once again, the formula may include conditioning
variables, for example \lstinline{z ~ x * y | g1 * g2}. The main
function for these graphics is \texttt{levelplot}.

The plotting of each panel is performed by the panel function, specified in
a high-level function call as the \texttt{panel} argument. Each
high-level \texttt{lattice} function has a default panel function,
although the user can create new Trellis displays with custom panel
functions.

\texttt{lattice} is a member of the recommended packages list so it is
commonly distributed with \textsf{R} itself. There are more than 250
packages depending on it, and the most important packages for our
purposes (\texttt{zoo}, \texttt{sp}, and \texttt{raster}) define
methods to display their classes using \texttt{lattice}.

\index{Packages!latticeExtra@\texttt{latticeExtra}} 

On the other hand, the \texttt{latticeExtra} package
\cite{Sarkar.Andrews2012} provides additional flexibility for the
somewhat rigid structure of the Trellis framework implemented in
\texttt{lattice}. This package complements the \texttt{lattice} with
the implementation of layers via the \texttt{layer} function, and
superposition of \texttt{trellis} objects and layers with the
\lstinline{+.trellis} function. Using both packages, you can define a
graphic with the formula interface (under the \texttt{lattice} model)
and overlay additional content as layers (following the
\texttt{ggplot2} model).

\subsection{ggplot2}
\label{sec:ggplot2}

\index{Packages!ggplot2@\texttt{ggplot2}} 

The \texttt{ggplot2} package \cite{Wickham2009} is an implementation
of the system proposed in \emph{The Grammar of Graphics}
\cite{Wilkinson1999}, a general scheme for data visualization that
breaks up graphs into semantic components such as scales and
layers. Under this framework, the definition of the graphic with
\texttt{ggplot2} is done with a combination of several functions
that provides the components, instead of the formula interface of
\texttt{lattice}.

With \texttt{ggplot2}, a graphic is composed of
\begin{itemize}
\item A dataset, \texttt{data}, and a set of mappings from variables
  to aesthetics, \texttt{aes}.
\item One or more layers, each composed of: a geometric object,
  \texttt{geom\_*}, to control the type of plot you create (points,
  lines, etc.); a statistical transformation, \texttt{stat\_*}; and a
  position adjustment (and optionally, additional dataset and
  aesthetic mappings).
\item A scale, \texttt{scale\_*}, to control the mapping from data to
  aesthetic attributes. Scales are common across layers to ensure a
  consistent mapping from data to aesthetics.
\item A coordinate system, \texttt{coords\_*}.
\item Optionally, a faceting specification, \texttt{facet\_*}, the
  equivalent of Trellis graphics with panels.
\end{itemize}

The function \texttt{ggplot} is typically used to construct a plot
incrementally, using the \texttt{+} operator to add layers to the
existing ggplot object.  For instance, the following code (equivalent to
the previous \texttt{lattice} example) uses \texttt{mtcars} as
the dataset, and maps the \texttt{mpg} variable on the x-axis and the
\texttt{wt} variable on the y-axis. The geometric object is the point
using the \texttt{cyl} variable to control the color. Finally, the
levels of the \texttt{am} variable define the panels of the graphic.

\lstset{language=R,numbers=none}
\begin{lstlisting}
ggplot(mtcars, aes(mpg, wt)) +
    geom_point(aes(colour=factor(cyl))) +
    facet_grid(. ~ am)
\end{lstlisting}
 
This package is increasingly popular, with a list of more than ninety
packages depending on it. On the other hand, few packages provide
method definitions based on \texttt{ggplot2} to display their
classes. In our context, only the \texttt{zoo} package defines the
\texttt{autoplot} function based on it.

\subsection{Comparison between lattice and ggplot2}
\label{sec:comparison}

Which package to choose is, for a wide range of datasets, a question
of personal preferences. You may be interested in a comparison
between them published in a series of blog
posts\footnote{http://learnr.wordpress.com/2009/06/28/ggplot2-version-of-figures-in-lattice-multivariate-data-visualization-with-r-part-1/}.
However, the major drawback of \texttt{ggplot2} is its considerably
slower speed when dealing with large datasets\footnote{Take a look at
  the time comparison published as the final result of the previous
  series of blog posts,
  http://learnr.files.wordpress.com/2009/08/latbook.pdf}, so you
should be cautious with large spatial and spatiotemporal data.

Consequently, most of the code in Part \ref{part:Time} contains
alternatives defined both with \texttt{lattice} and with
\texttt{ggplot2}. However, because of the speed problem and the
absence of \texttt{ggplot2} functions in the corresponding packages,
only a minor fraction of the code in Parts \ref{cha:Spatial} and
\ref{cha:Spatio-Time} contains graphics defined with \texttt{ggplot2}.
  

\section{Packages}
\label{sec:introduction-packages}

Throughout the book, several \textsf{R} packages are used. All of them
are available from \textsf{CRAN}, and you must install them before
using the code. Most of them are loaded at the start of the code of
each chapter, although some of them are loaded later if they are used
only inside optional sections (marked with \floweroneleft). You should
install the last version available at \textsf{CRAN} to ensure correct
functioning of the code.

Although the introductory chapter of each part includes a section with
an outline of the most relevant packages, some of them deserve to be
highlighted here:

\begin{itemize}

\item \texttt{zoo} \cite{Zeileis.Grothendieck2005} provides
  infrastructure for time series using arbitrary classes for the time
  stamps (Section \ref{sec:zoo}).

\item \texttt{sp} \cite{Pebesma2012} provides a coherent set of
  classes and methods for the major spatial data types: points, lines,
  polygons, and grids (Section \ref{sec:sp}). \texttt{spacetime}
  \cite{Pebesma2012} defines classes and methods for spatiotemporal
  data, and methods for plotting data as map sequences or multiple
  time series (Section \ref{sec:spacetime}).

\item \texttt{raster} \cite{Hijmans2013} is a major extension of
  gridded spatial data classes. It provides a unified access method to
  different raster formats, permitting large objects to be analyzed
  with the definition of basic and high-level processing functions
  (Sections \ref{sec:raster} and \ref{sec:rasterST}). \texttt{rasterVis}
  \cite{Perpinan.Hijmans2013} provides enhanced visualization of
  raster data with methods for spatiotemporal rasters (Sections
  \ref{sec:rasterVis} and \ref{sec:rastervisST}).

\item \texttt{gridSVG} \cite{Murrell.Potter2013} converts any grid
  scene to an \textsf{SVG} document. The \lstinline{grid.hyperlink}
  function allows a hyperlink to be associated with any component of
  the scene, the \lstinline{grid.animate} function can be used to
  animate any component of a scene, and the \lstinline{grid.garnish}
  function can be used to add \textsf{SVG} attributes to the
  components of a scene. By setting event handler attributes on a
  component, plus possibly using the \lstinline{grid.script} function
  to add \textsf{JavaScript} to the scene, it is possible to make the
  component respond to user input such as mouse clicks.

\end{itemize}



\section{Software Used to Write This Book}
\label{sec:software-book}

This book has been written using different computers running Debian
GNU Linux and using several gems of open-source software:
\begin{itemize}
\item \textsf{org-mode} for authoring text and code
  \cite{Schulte.Davison.ea2012}.
\item \textsf{R} \cite{RDevelopmentCoreTeam2013} with \textsf{Emacs
    Speaks Statistics} \cite{Rossini.Heiberger.ea2004}.
\item \LaTeX{} with AUC\TeX{} to produce the final document.
\item \textsf{GNU Emacs} as development environment.
\end{itemize}

\section{About the Author}
\label{sec:aboutMe}

During the past 15 years, my main area of expertise has been
photovoltaic solar energy systems, with a special interest in solar
radiation.

Initially I worked as an engineer for a private company and I was
involved in several commercial and research projects. The project
teams were partly integrated by people with low technical skills who
relied on the input from engineers to complete their work. I learned
how a good visualization output eased the communication process.

Now I work as a professor and researcher at the university. Data
visualization is one of the most important tools I have available. It
helps me embrace and share the steps, methods, and results of my
research. With students, it is an inestimable partner in helping them
understand complex concepts.

I have been using \textsf{R} to simulate the performance of
photovoltaic energy systems and to analyze solar radiation data,
both as time series and spatial data. As a result, I have
developed packages that include several graphical methods to deal
with multivariate time series (namely, \texttt{solaR}
\cite{Perpinan2012b}) and space-time data (\texttt{rasterVis}).

\section{Acknowledgments}
\label{sec:acknow}

Writing a book is often described as a solitary activity. It is
certainly difficult to write when you are with friends or spending
time with your family,... although with three little children at home
I have learned to write prose and code while my baby wants to learn
typing and my daughters need help to share a family of dinosaurs.

Seriously speaking, solitude is the best partner of a writer. But when
I am writing or coding I feel I am immersed in a huge collaborative
network of past and present contributors. Piotr Kropotkin described it
with the following words \cite{Kropotkin1906}:

\begin{quote}
  Thousands of writers, of poets, of scholars, have laboured to
  increase knowledge, to dissipate error, and to create that
  atmosphere of scientific thought, without which the marvels of our
  century could never have appeared. And these thousands of
  philosophers, of poets, of scholars, of inventors, have themselves
  been supported by the labour of past centuries. They have been
  upheld and nourished through life, both physically and mentally, by
  legions of workers and craftsmen of all sorts.
\end{quote}

And Lewis Mumford claimed \cite{Mumford1934}:

\begin{quote}
  Socialize Creation! What we need is the realization that the
  creative life, in all its manifestations, is necessarily a social
  product.
\end{quote}

I want to express my deepest gratitude and respect to all those women
and men who have contributed and contribute to strengthening the
communities of free software, open data, and open science. My special
thanks go to the people of the \textsf{R} community: users, members
of the \textsf{R} Core Development Team, and package developers.

With regard to this book in particular, I would like to thank John
Kimmel for his constant support, guidance, and patience.

Last, and most importantly, thanks to Candela, Marina, and Javi, my
crazy little shorties, my permanent source of happiness, imagination, and
love. Thanks to María, \emph{mi amor, mi cómplice y todo}.

%%% Local Variables:
%%% mode: LaTex
%%% TeX-master: "../main.tex"
%%% End: 
